\documentclass[11pt]{article}
\usepackage{eacl2017}
\usepackage{times}
\usepackage{url}
\usepackage{latexsym}
\usepackage{natbib}
\usepackage{array}
\usepackage{graphicx}
\usepackage{algorithm}
\usepackage{algpseudocode}
\usepackage{mathtools}
\usepackage{caption}
\usepackage{chngcntr}

\eaclfinalcopy

\title{Operator Prediction through Sentence Simplication for Arithmetic Word Problems}

\author{First Author \\
  Affiliation / Address line 1 \\
  Affiliation / Address line 2 \\
  Affiliation / Address line 3 \\
  {\tt email@domain} \\\And
  Second Author \\
  Affiliation / Address line 1 \\
  Affiliation / Address line 2 \\
  Affiliation / Address line 3 \\
  {\tt email@domain} \\}

\date{}

\begin{document}
\maketitle
\begin{abstract}
 This paper presents a sentence simplification approach in learning to solve arithmetic word problems. The approach performs a thorough analysis to map each sentence in the word problem to a simplified sentence . The objective of our approach is to use the basic properties of the language to simplify sentences and then classify the simplified sentences to their operators. We simplify the sentence until it represents a single operation. Using the predicted operators for each simplified sentence we build an equation and solve the problem. We train our classifier on MAWPS (MAth Word Problem Respository) dataset and achieve an accuracy of 83\%.  Experimental results show that our method outperforms existing systems, achieving state of the art performance on benchmark datasets of arithmetic word problems.
\end{abstract}

\section{Introduction}
Interpreting a sentence representing a single mathematical operation is relatively easier than interpreting a sentence having multiple mathematical operations. For example it will be difficult to extract quantities from the sentences in the actual question. However, it becomes much simpler to extract information useful for the equation from the simplified sentences in order to solve the problem.

\begin{table}[h!]
\centering
\begin{tabular}{ | m{25em} | }
\hline
\textbf{Example 1:}\\
\hline
A spaceship traveled 0.5 light-year from Earth to Planet X and 0.1 light-year from Planet X to Planet Y. How many light-years did the spaceship travel in all ?\\
\hline
A spaceship traveled 0.5 light-year from Earth to Planet X \\
\hline
A spaceship traveled 0.1 light-year from Planet X to Planet Y \\
\hline
\end{tabular}
\caption{Example Arithmetic Word Problem.}
\label{figure:1}
\end{table}
			
To simplify the word problem, we execute a set of rules on each sentence so that it possibly has multiple simplified sentences. Our goal here is to have a single operation in each simplified sentence. To solve the problem, we extract an equation using the coherent set of simplified sentences and their predicted operators. 

\end{document}